\documentclass[man,donotrepeattitle,letter]{apa6}

\usepackage[american]{babel}
\usepackage{csquotes}
\usepackage[style=apa,sortcites=true,sorting=nyt,backend=biber]{biblatex}
\usepackage{float}
\floatstyle{boxed}
\restylefloat{figure}
\usepackage{graphicx}
\usepackage{lipsum}
% \usepackage[utf8]{inputenc}
% \DeclareUnicodeCharacter{00A0}{ }
\DeclareLanguageMapping{american}{american-apa}
\addbibresource{main.bib}


\title{GEOG281 Final Examination -- Responses on the Korean Wave \& Hong Kong Chinese Emmigration to Vancouver}
\shorttitle{GEOG281 Final Exam}

\author{Evan Louie -- 72210099}

\affiliation{University of British Columbia}

\abstract{Both the development of the Korean Wave and the development of global neoliberal hegemony in Vancouver and subsequent influx of Hong Kong Chinese immigration were major events in the development of the Pacific Rim network.  In this paper, articles written by Prof. Sino Nam and Prof. Katharyne Mitchell are examined and reflected upon.  With the former looking at the occurence of the Korean Wave from a post-Fordist global economy and the latter describing the happenings between Vancouver and Hong Kong from the perspective of social liberalism and neoliberalism.}

\keywords{GEOG281, final, examination, Pacific Rim, Pacific-Asia, east Asia, south-east Asia, Japan, China, Korea, United States, relations, formation, korean wave, k-pop, psy, gangnam style, vancouver, immigration, hong kong, china, inflation}

\authornote{Intended for TA \textbf{Kelsey Johnson} and GEOG281 affiliated staff}
\begin{document}
\maketitle

\tableofcontents
\newpage

\section{The Cultural Political Economy of the Korean Wave in East Asia}
Over the past two decades, the Korean Wave, a neologism coined by Beijing journalists denoting China's increasing appetite for Korean cultural exports, has quickly gained traction and has made Korean pop culture one of Korea's chief exports, making it ``Asia's foremost trendsetter''.  Rivaling many Western nations such as the United States and United Kingdom as exporters of popular culture, the Korean Wave has enabled Korea to break into an industry that the United States has had a near monopolistic hold of for nearly a century (Leong M, 2014).

\subsection{Just another Hollywood?}
In \textit{The Cultural Political Economy of the Korean Wave}, University of North Florida professor, Sino Nam, posits the question of whether or not the Korean Wave is ``... a forceful antithesis to US-centric globalization, or is it a slipshod copy of Hollywood at the periphery of global cultural geography?'' (2013). Addressing the question from the point of view of ``cultural globalization in the neoliberal post-Fordist global economy'' (p. 210), Nam does not answer this question directly, but draws several derivative conclusions that do.  Believing the Korean Wave to be more akin to Hollywood than its antithesis, Korea ``reproduces the post-Fordist global economy with its diversified, Korean-spiced media artifacts'' (p. 227). The Korean Wave has ``less to do with democratic cultural globalization in its fullest sense than it does with the discursive, strategic response to the realignment of post-Fordist global cultural economy.'' (p. 229). With this in mind, it becomes clear that Nam believes the Korean Wave to be, while not a definitive pushback against US-centric globalization, a response to fill in niches that Western US-centric media does not.  By emulating the Hollywood style of production and marketing, while adding a distinct South Korean sense of style and culture, the Korean Wave becomes a ``nearly perfect embodiment... of cultural globalization'' (p. 229).

\subsection{\textit{Gangnam Style} -- K-Pop Goes Global}
Released on July 15, 2012, Psy's \textit{Gangnam Style} made headlines around the world as its popularity quickly skyrocketed and became YouTube's most watched video of all time. Depicting a goofy man and his antics while trying to act as if he is from the Gangnam district (a rich, upper-class district in South Korea), \textit{Gangnam Style} represents a satirical representation of what atypical K-Pop usually conveys.  A long stride away from the traditionally polished, beautiful boy/girl groups normally associated with K-Pop, Psy's departure from normative Korean Wave materials makes for an interesting case in terms of how the Korean Wave spreads and the means in which it does so.  Originally written to be a satirical look at K-Pop and the social class revolving around Gangnam, it is a wonder how the song/video, intended to be an in-joke for existing fans of K-Pop and the Korean Wave, became as popular as it did.  If one was to look at the \textit{Gangnam Style} phenomena from the perspective that Nam posits about the Korean Wave, it simply does not fit. However, it does fit into Nam's overall narrative of globalization in post-Fordist economies. Positing the Korean Wave to be a ``nearly perfect embodiment of, rather than the antithesis of, cultural globalization'' (p. 229), Nam believes the Korean Wave to be a form of popular culture which fills in the niches left behind from local/regional and mainstream Western medias.  With the foundation of Gangnam Style's production being satire of the Korean Wave, it become apparent that the phenomena should not be attributed to the Korean Wave, but more so to an underlying sense of globalization and international connectivity enabled by technologies such as YouTube.  The popularity of \textit{Gangnam Style} can be much more attributed to the song having a catchy musical composition akin to those of popular party songs at the time by American artists such as LMFAO and an eye-catching visual style wherein the inherit unique goofiness makes strong viewer retention; making for a very well executed viral video.

\subsection{Korean Wave vs. Classical Investment}
Being a cultural export, the Korean Wave is a soft power that is difficult to concisely measure or analyze compared to that of Korea's other regionalization projects within Pacific Asia.  With a burgeoning economy and globally known brands such as Samsung and Hyundai, it is clear that South Korea's material exports are indeed successful and have found strong markets both locally and abroad. However, with their main exports being technologically based, South Korea has had difficulties creating markets within Southeast Asia, where the economic disparities between Asian countries becomes ever more apparent. Unlike, China, the U.S, and Japan (South Korea's top exporting partners), the countries within Southeast Asia do not have strongly developed economies and match more so to Newly Industrialized Countries (NIC).

Officially beginning dialogues with ASEAN in 1989, economic priorities had always lay inwards towards the peninsula, followed by Northeast Asia, then finally, broader regions and beyond (Balbina, 2012).  However, as South Korea's economy has continued to grow, it has reached a point in which it may actively begin regional and global development.  With notable investments in Laos in the form of dam development along the Mekong River, South Korea is going about an atypical form of loan investment within Southeast Asia. Running opposite their approach with the Korean Wave, South Korea is attempting to emulate what the United States did with them in the post Korean War era.  By investing in the infrastructure of developing countries, South Korea both fosters relations between the two and will also, hopefully, see a return on investment in the future.  If one was to compare this classical style of foreign investment with that of a cultural export such as the Korean Wave, it becomes apparent that core difference is target demographic.  The Korean Wave is a vehicle in which Korean popular culture can flourish abroad. With the main consumers of popular culture being strong consumerist countries, the main economic return on investment from the Korean Wave is the ability to market their exports.  No matter how well marketed their technological exports are within the Southeast Asian region, the region will not yield a noticeable consumer market share until becoming economically stronger.  As such, by using a more classical style of foreign investment, South Korea is able to establish a cultural presence within developing or Newly Industrialized Countries (NIC), just as it does with economically powerful countries through the Korean Wave.

\section{Vancouver Goes Global}
The 1980's saw a decisive shift from an ethic of contemporary social liberalism to a stance of laissez-faire style neoliberalism in Vancouver.  Playing host to one of the most significant waves of immigration in Canadian history, Hong Kong Chinese (HKC) with looming thoughts of the upcoming July 1, 1997 returning of British Hong Kong to China, began to flood Vancouver in hopes of escaping the grasp of communist China; and with them, one of the largest influxes of foreign capital Vancouver had ever scene. In the article \textit{Vancouver Goes Global}, Professor of Geography at the University of Washington, Katharyne Mitchell discusses this occurrence from the perspective of neoliberalism and Vancouver's shift into such ideology.

\subsection{From Hong Kong to Vancouver -- \textit{``Hongcouver''}}
Leading into the 1980's, the decades prior showed an ideologically different version of Vancouver than what was would eventually come to be.  With the socially and economically left New Democratic Party holding provincial power throughout the 1970's, the quick rise of conservative neoliberal hegemony within Vancouver and the election of the Conservative leader, Bill Bennett, as BC Premier in 1980 changed Vancouver's political landscape greatly. This shift would eventually lead to the pushing of Vancouver onto the global stage, as it was subsequently publicized as an economically safe and viable place for Hong Kong citizens to emigrate and establish residency. HKC came en masse, bringing with them both large amounts of Hong Kong capital and inflation.  As HKC continued to poor in, Vancouver's housing sector quickly rose, property was being bought for exorbitantly high prices unseen in Vancouver's market and ``monster houses'' quickly began to be built in traditionally British style neighborhoods.  The subsequent pushback from native Vancouver citizens was inevitable.  As discussions of the merits of classical social liberalism versus egalitarian style neoliberalism became main points of argument among Vancouver citizens, racial tensions also began to rise. Long time Vancouver residence had their houses and property bought out from under them and all of the foreigners doing this happened to be HKC.  Whether or not the racism towards the HKC was indeed stemming from innate cultural racism or a byproduct of generic hate towards an external group disrupting the living standards cannot be determined. But, as Mitchell states, ``controlling racial definitions and explanations... became a key tactic in the struggle to maintain the moral high ground in the battle over interpreting the changing urban landscape [of Vancouver]'' (p. 84).

\subsection{Socal Liberalism vs. Neoliberalism}
In the socially progressive city of Vancouver, maintaining the moral high ground stays a point of importance amongst political debate. With the influx of HKC to Vancouver, debates between the older social liberalism and the new global neoliberalism came into conflict. With a population that maintained a social liberal stance up until the 1980's and even thereafter, the influx of HKC caused conflict between social liberals and neoliberals in both ideology and, in the case of social liberals, conflict within the ideology itself between what the ideological subscribers wanted and how the ideology was defined.

As the neoliberal agenda promoted the immigration of HKC into Vancouver, long-time residence began to show distain for such decisions as inflation and clashes in culture began to show.  Oddly enough, social liberals were left with little recourse in terms of actual political or ideological rebuttal. Positing the notion of ``territory'' and the formation of societies based around space, Mitchell reiterates the correlation between the formation of a political identity and the constitution of territory.  With both social liberalism and neoliberalism denouncing the importance of space/class in the formation of societies, social liberals were left with little means to deflect discourse revolving around such notions of space.  As ``long-term urban residence could attack a neoliberal urban agenda for its mercenary motives and lack of human warmth or sense of territorial allegiance, promoters could uphold it as a global agenda untainted by local, racial, or class antagonisms'' (p. 86).  As Vancouver social liberals did indeed want stricter maintaining of social class and space, admitting so, would validate the notion of the importance of space and its role in society; thus run counter to the social liberal ideology.

\subsection{Race \& Class -- Shaping The Debate}
The HKC did not only bring a new culture with them to Vancouver, they also brought in incredible amounts of Hong Kong capital; amounts that even much of Vancouver's upper class could not compete with. As Hong Kong had been a British colony since 1841, the Chinese from Hong Kong were very different from those from the mainland. Rich and fairly Westernized from generations of British colonization, Hong Kong emigrants to Vancouver came expecting ``to be first-class citizens, they wanted to live in the best neighborhoods, wanted the best schools for their kids.'' (Canada.com, 2007) As such, displacement of long-time Vancouver citizens began to occur in the historically upper class bastions of Vancouver, most noticeably, Kerrisdale. Of the 3,225 apartment renters in the Kerrisdale area, 1,595 of them were above the age of 65 (p. 75). Among these people, the vast majority were upper-class women who sold their homes and began renting after being widowed. The subsequent evictions of these older white women became a rallying point to native Vancouver citizens, symbolizing ``the negative effects of foreign investment'' (p. 76). With historic leaning towards the capitalistic side (p. 80), the residences of Kerrisdale were now seeing the negative aspects of such an ideology.

With groups within the opposition of HKC immigration showing ``indisputable evidence of racist agendas'' (p. 82), criticisms made by the typically white opposition were easily made to appear suspect.  What was fundamentally a clash of classes became inescapably linked to race; thus making debate by localists near impossible. As race and class became a complicated and dangerous point of discussion in shaping local struggles over housing and urban life in Vancouver, ``controlling [of] racial definitions and explanations'' was key in maintaining any sort of debate about the urban redevelopment; whether by the Vancouver localists or the neoliberal government.



\nocite{*}
\printbibliography

\newpage
[This page is intentionally left blank.]

\end{document}

%
% Please see the package documentation for more information
% on the APA6 document class:
%
% http://www.ctan.org/pkg/apa6
%
